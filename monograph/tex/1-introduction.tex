\chapter{Introdução}
\label{cap:introducao}

% ---------------------------------------------------------------------

\section{Motivação e objetivo}

Hoje em dia, não há como negar que o mercado de \emph{games} é um fenômeno mundial,
gerando mais de US\$ 91 bilhões em 2016 \citep{gameMarket:16}. Comparado aos
primeiros jogos, comercializados no início da década de 1970
\citep{gameMarketOrigin}, a evolução em diversas áreas da computação permitiu
grandes avanços nos jogos criados. Inclui-se nisso a evolução dos computadores por
conta da \emph{Lei de Moore} \citep{moore}, permitindo processamento mais rápido;
\emph{games} em 3D e gráficos cada vez mais sofisticados e realistas devido à
\emph{Computação Gráfica}; e adversários sofisticados e de raciocínio rápido com a
\emph{Inteligência Artificial}.

Juntamente com os próprios jogos, as tecnologias usadas para desenvolvê-los também
tiveram progressos. Em especial, temos as \emph{game engines}, que podem ser
descritas como \textquotedblleft \emph{frameworks} voltados especificamente para a
criação de jogos\,\textquotedblright\:\citep{gameEngine:13}. Oferecem diversas
ferramentas para acelerar o desenvolvimento do jogo, como maior facilidade na
manipulação gráfica e bibliotecas prontas para tratar colisões entre objetos. Como
eficiência é um fator essencial em um jogo, as \emph{engines} costumam ter sua base
construída em linguagens rápidas e compiladas, como C e C++.

Focaremos em uma \emph{game engine} em particular, \emph{Godot} \citep{godot}. O
principal motivo de ter sido escolhida é por ser um \emph{software} de código
aberto, o que permite a qualquer pessoa baixar seu código fonte e fazer
modificações. Em especial, a \emph{engine} permite a criação de novos módulos para
adicionar a ele novas funcionalidades.

Este trabalho visa a criar um novo módulo para \emph{Godot}. Tal extensão adicionará
funções simples de reconhecimento de voz, algo ainda inexistente no \emph{software}.
Feito isso, a nova funcionalidade será demonstrada em um jogo simples criado nessa
\emph{engine}.

% ---------------------------------------------------------------------

\section{Organização do trabalho}
% TODO: Link chapter numbers to respective chapters!

O capítulo 2 contém pesquisas e buscas por uma biblioteca de código aberto que faça
reconhecimento de voz eficientemente. O capítulo 3 consiste em integrar a biblioteca
encontrada ao \emph{Godot}, expandindo suas funcionalidades. No capítulo 4,
mostram-se os passos realizados para criar um jogo que demonstre a capacidade do
novo módulo. O capítulo 5 apresenta as conclusões do trabalho. Por fim, há uma parte
subjetiva contendo a apreciação pessoal do TCC e uma descrição das matérias que mais
ajudaram no desenvolvimento do projeto.
