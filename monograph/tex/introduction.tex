\chapter{Introdução}
\label{cap:introduction}

% ---------------------------------------------------------------------

\section{Motivação e objetivo}

Hoje em dia, não há como negar que o mercado de \textit{games} é um fenômeno mundial, gerando mais de US\$ 91 bilhões em 2016 \citep{gameMarket:16}. Comparado aos primeiros jogos, comercializados no início da década de 1970 \citep{gameMarketOrigin}, a evolução em diversas áreas da computação permitiu grandes avanços nos jogos criados. Inclui-se nisso a evolução dos computadores por conta da \textit{Lei de Moore} \citep{moore}, permitindo processamento mais rápido; \textit{games} em 3D e gráficos cada vez mais sofisticados e realistas devido à Computação Gráfica; e adversários sofisticados e de raciocínio rápido com a Inteligência Artificial.

Junto aos próprios jogos, as tecnologias usadas para desenvolvê-los também tiveram progressos. Em especial, temos as \emph{game engines}, que podem ser descritas como \textquotedblleft \textit{frameworks} voltados especificamente para a criação de jogos\,\textquotedblright\:\citep{gameEngine:13}. Elas oferecem diversas ferramentas para acelerar o desenvolvimento de um jogo, como maior facilidade na manipulação gráfica e bibliotecas prontas para tratar colisões entre objetos. Além disso, como eficiência é um fator essencial para manter um bom valor de FPS (\textit{Frames per Second}), as \textit{engines} costumam ter sua base construída em linguagens rápidas e compiladas, como C e C++.

Focaremos em uma \textit{game engine} em particular, \textbf{\emph{Godot}} \citep{godot}. O principal motivo de ter sido escolhida é por ser um \textit{software} de código aberto, o que permite a qualquer pessoa baixar seu código fonte e fazer modificações. Em especial, a \textit{engine} permite a criação de novos módulos para adicionar a ele novas funcionalidades.

Este trabalho visa a criar um novo módulo para \textit{Godot}. Tal extensão adicionará funções simples de reconhecimento de voz, algo ainda inexistente no \textit{software}. Feito isso, a nova funcionalidade será demonstrada em um jogo simples criado nessa \textit{engine}.

% ---------------------------------------------------------------------

\section{Organização do trabalho}

% TODO: Organize this later
O capítulo \ref{cap:speech-libs} contém pesquisas e buscas por uma biblioteca de código aberto que faça reconhecimento de voz eficientemente. O capítulo \ref{cap:godot-module} consiste em integrar a biblioteca encontrada ao \emph{Godot}, expandindo suas funcionalidades. No capítulo \ref{cap:game-sample}, mostram-se os passos realizados para criar um jogo que demonstre a capacidade do novo módulo. O capítulo \ref{cap:conclusion} apresenta as conclusões do trabalho.

Por fim, há uma parte subjetiva contendo a apreciação pessoal do TCC e uma descrição das matérias que mais ajudaram no desenvolvimento do projeto.
