\chapter{Introdução}
\label{cap:introduction}

% ---------------------------------------------------------------------

\section{Motivação e objetivo}

Hoje em dia, não há como negar que o mercado de \textit{games} é um fenômeno mundial, gerando mais de US\$ 91 bilhões em 2016 \citep{gameMarket:16}. Comparado aos primeiros jogos, comercializados no início da década de 1970 \citep{gameMarketOrigin}, a evolução em diversas áreas da computação permitiu grandes avanços nos jogos criados. Inclui-se nisso a evolução dos computadores por conta da \textit{Lei de Moore} \citep{moore}, permitindo processamento mais rápido; \textit{games} em 3D e gráficos cada vez mais sofisticados e realistas devido à Computação Gráfica; e adversários sofisticados e de raciocínio rápido com a Inteligência Artificial.

Junto aos próprios jogos, as tecnologias usadas para desenvolvê-los também tiveram progressos. Em especial, temos as \emph{game engines}, que podem ser descritas como \textquotedblleft \textit{frameworks} voltados especificamente para a criação de jogos\,\textquotedblright\:\citep{gameEngine:13}. Elas oferecem diversas ferramentas para acelerar o desenvolvimento de um jogo, como maior facilidade na manipulação gráfica e bibliotecas prontas para tratar colisões entre objetos. Além disso, como eficiência é um fator essencial para manter um bom valor de FPS (\textit{Frames per Second}), as \textit{engines} costumam ter sua base construída em linguagens rápidas e compiladas, como C e C++.

Focaremos em uma \textit{game engine} em particular, \textbf{\emph{Godot}} \citep{godot}. O principal motivo de ter sido escolhida é por ser um \textit{software} de código aberto, o que permite a qualquer pessoa baixar seu código fonte e fazer modificações. Em especial, a \textit{engine} permite a criação de novos módulos para adicionar a ele novas funcionalidades.

Este trabalho visa a criar um novo módulo para \textit{Godot}. Tal extensão adicionará funções simples de reconhecimento de voz, algo ainda inexistente no \textit{software}. Feito isso, a nova funcionalidade será demonstrada em um jogo simples criado nessa \textit{engine}.

% ---------------------------------------------------------------------

\section{Organização do trabalho}

O capítulo \ref{cap:speech-recognition} aborda resumidamente reconhecimento de voz através de um olhar teórico.\iffalse A seguir, no capítulo \ref{cap:hmm}, apresenta-se uma forma de realizar reconhecimento de voz por meio do \textit{Modelo Oculto de Markov}.\fi

No capítulo \ref{cap:speech-libs}, são realizados os primeiros passos para a concretização deste trabalho; busca-se a melhor biblioteca de reconhecimento de voz que possa ser usada no módulo. A biblioteca escolhida, \textit{Pocketsphinx}, é estudada no capítulo \ref{cap:pocketsphinx}.

A arquitetura do \textit{Godot} é apresentada no capítulo \ref{cap:godot} a fim de se entender a lógica por trás da construção do módulo de reconhecimento de voz no capítulo \ref{cap:stt-module}. O capítulo \ref{cap:color-clutter} apresenta a criação de jogo simples, feito na própria \textit{game engine}, para demonstrar o módulo em funcionamento e suas capacidades.

% TODO: Check where the subjective chapters will be placed
O capítulo \ref{cap:conclusion} apresenta as conclusões do trabalho. Por fim, há uma parte subjetiva contendo a apreciação pessoal do TCC e uma descrição das matérias que mais ajudaram no desenvolvimento do projeto.
