\chapter{Pocketsphinx}
\label{cap:pocketsphinx}

Neste capítulo, analisaremos mais a fundo a biblioteca \textit{Pocketsphinx}, incluindo seu funcionamento, instruções para usá-la de forma básica e passos para compilação a partir do código fonte.

Supõe-se que o usuário esteja usando um sistema operacional \textit{Unix}, e que possua acesso a privilégios administrativos para a realização de alguns passos. Recomenda-se que o leitor possua um microfone à disposição no computador, podendo ser embutido ou externo, para melhor aproveitamento.

Todas as instruções e comandos apresentados foram originalmente realizados no sistema \texttt{Ubuntu 16.04 LTS, 64-bit} do autor.

% ---------------------------------------------------------------------

\section{Funcionamento}

O funcionamento das bibliotecas do projeto \textit{CMUSphinx}, incluindo-se a \textit{Pocketsphinx}, pode ser resumido por três grandes passos:

\begin{itemize}
\item A configuração inicial de arquivos a serem usados pela biblioteca, como o dicionário.
\item A captura de áudio de voz, separando-a em \textit{utterances}.
\item A busca, para cada \textit{utterance}, da melhor combinação de palavras do dicionário que se assemelhe a ela.
\end{itemize}

Definimos, abaixo, alguns conceitos numa ordem que nos proporcione um melhor entendimento das etapas descritas.

% ---------------------------------------------------------------------

\subsection{Fonema}

Um \textbf{fonema} é a menor unidade de som em uma língua.

O leitor poderia pensar que uma palavra é uma sequência de fonemas, mas tal definição esconde diversas complexidades: há sons que surgem na transição entre palavras e variantes linguísticas na pronúncia do falante, por exemplo. Devido a isso, surgem termos como \textit{difonemas} e \textit{trifonemas}, que tratam de fonemas consecutivos para levar em conta o contexto em que o som é captado.

% ---------------------------------------------------------------------

\subsection{Vetor de características}

Em aprendizado de máquina, uma \textbf{característica} é uma quantidade que descreve algum exemplo.

No contexto de reconhecimento de voz, CMUSphinx divide as \textit{utterances} em quadros (\textit{frames}) de aproximadamente 10 ms de comprimento. Através de uma função complexa, extraem-se 39 números -- características -- para representar a \textit{utterance}; juntos, eles formam o \textbf{vetor de características}.

% ---------------------------------------------------------------------

\subsection{Modelo}

Um \textbf{modelo} é uma simplificação, onde reduz-se o que se quer modelar às suas características mais importantes. Neste caso, falamos de um modelo de reconhecimento de voz: como tratar as transições entre os quadros em que se divide o áudio capturado?

A solução encontrada pelo projeto CMUSphinx foi utilizar o Modelo Oculto de Markov (\textit{Hidden Markov Model}, ou HMM, conforme visto na seção \ref{cap:hmm}) para tratar a fala gravada como uma sequência de estados que transitam entre si com certa probabilidade.

Buscam-se os estados do HMM que levam à maior probabilidade no vetor de características. Para isso, três modelos, alimentados à biblioteca na forma de arquivos externos, são usados:

\begin{itemize}
\item \textbf{Modelo acústico}: Conjunto de arquivos que contém propriedades acústicas para detectores de fonemas. Define os vetores de características mais prováveis para cada unidade de som, além de determinar a criação de uma sequência de fonemas para um dado contexto. Este modelo costuma vir na forma de vários arquivos. Possui alta dependência com a língua na qual se realiza o reconhecimento de voz.

\item \textbf{Dicionário fonético}: Arquivo texto responsável por mapear palavras em fonemas, que devem existir segundo o modelo acústico. Um mapeamento perfeito é praticamente impossível; devido a variantes linguísticas e outros fatores, não há como adicionar todas as diferentes formas de se pronunciar uma palavra.

Um exemplo de uma linha de um dicionário em inglês seria:

\begin{center}
\texttt{yellow Y EH L OW}
\end{center}

\item \textbf{Modelo de linguagem}: Arquivo que formaliza uma sintaxe para a linguagem a ser reconhecida. Sua principal finalidade é diminuir o espaço de busca nas palavras, descartando-se palavras improváveis no áudio capturado e melhorando a acurácia.
\end{itemize}

% ---------------------------------------------------------------------

\subsection{Palavras-chave}

Dentre várias formas diferentes de busca, \textit{CMUSphinx} também oferece suporte para reconhecimento de voz por palavras-chave. Ao invés de usar um modelo de linguagem, fornece-se à biblioteca um arquivo de palavras ou frases a qual se quer detectar, juntamente com um limiar de detecção. Qualquer som capturado que não se encaixar no arquivo ou cujo limiar calculado for baixo demais será descartado.

Um exemplo de linha no arquivo de palavras-chave está a seguir: cada palavra deve vir seguida de seu limiar. Destaca-se que este valor deve vir isolado entre caracteres \texttt{``/''}.

\begin{center}
\texttt{yellow /1e-6/}
\end{center}

% ---------------------------------------------------------------------

\section{Compilação}
\label{sphinxCompile}

Apresentamos instruções, em \textit{Bash}, para baixar e compilar a biblioteca \textit{Pocketsphinx}. Os passos foram baseados nas instruções em \citep{pocketsphinxInstall}.

Antes de começar, instale as seguintes dependências em seu sistema:

\begin{center}
\footnotesize\texttt{gcc, automake, autoconf, libtool, bison, swig, python-dev, pulseaudio}
\end{center}

Em um sistema \emph{Ubuntu}, por exemplo, digitaria-se no terminal:

\begin{lstlisting}[language=Bash]
$ sudo apt-get install gcc automake autoconf libtool bison swig \
  python-dev pulseaudio
\end{lstlisting}

% ---------------------------------------------------------------------

\subsection{Pacote \textit{Sphinxbase}}
\label{sphinxbaseCompile}

O pacote \textbf{Sphinxbase} oferece funcionalidades comuns a todos os projetos \textit{CMUSphinx}. Siga as instruções abaixo para compilá-lo.

\begin{enumerate}
\item Clone o repositório do \textit{Sphinxbase}.

\begin{lstlisting}[language=Bash]
$ git clone https://github.com/cmusphinx/sphinxbase
\end{lstlisting}

\item Dentro do diretório \texttt{sphinxbase/} criado pelo passo anterior, execute o \textit{script} \texttt{autogen.sh} para gerar o arquivo \texttt{configure}:

\begin{lstlisting}[language=Bash]
$ ./autogen.sh
\end{lstlisting}

\item Execute o \textit{script} \texttt{configure} criado no último passo:

\begin{lstlisting}[language=Bash]
# Padrão
$ ./configure

# Plataformas sem aritmética de ponto flutuante
$ ./configure ---enable-fixed ---without-lapack
\end{lstlisting}

Note que qualquer dependência ausente no sistema (por exemplo, o pacote \texttt{swig}) será notificada ao usuário neste passo. Se a execução ocorrer sem problemas, um \texttt{Makefile} será gerado.

\item Compile o \textit{Sphinxbase} através do \texttt{Makefile}:

\begin{lstlisting}[language=Bash]
$ make
\end{lstlisting}

\end{enumerate}

% ---------------------------------------------------------------------

\subsection{Pacote \textit{Pocketsphinx}}
\label{pocketsphinxCompile}

O pacote \textbf{Pocketsphinx} contém as funcionalidades de reconhecimento de voz em si que nos interessam para este trabalho. Siga as instruções abaixo para compilá-lo.

\begin{enumerate}
\item Clone o repositório do \textit{Pocketsphinx}, o que criará o diretório \texttt{pocketsphinx/}.

\begin{lstlisting}[language=Bash]
$ git clone https://github.com/cmusphinx/pocketsphinx
\end{lstlisting}

\item Certifique-se que as pastas \texttt{sphinxbase/} e \texttt{pocketsphinx/} estejam no mesmo diretório, pois \textit{Pocketsphinx} usa o caminho \texttt{../} para procurar pelo pacote \textit{Sphinxbase}.

\item Dentro do diretório \texttt{pocketsphinx/}, execute o \textit{script} \texttt{autogen.sh} para gerar o arquivo \texttt{configure}:

\begin{lstlisting}[language=Bash]
$ ./autogen.sh
\end{lstlisting}

\item Execute o \textit{script} \texttt{configure} criado no último passo:

\begin{lstlisting}[language=Bash]
$ ./configure
\end{lstlisting}

Note que qualquer dependência ausente no sistema será notificada ao usuário neste passo. Se a execução ocorrer sem problemas, um \texttt{Makefile} será gerado.

\item Compile o \textit{Pocketsphinx} através do \texttt{Makefile}:

\begin{lstlisting}[language=Bash]
$ make
\end{lstlisting}

\end{enumerate}

% ---------------------------------------------------------------------

\subsection{Teste de verificação}

Para verificar se a compilação feita nas subseções \ref{sphinxbaseCompile} e \ref{pocketsphinxCompile} ocorreu corretamente, recomenda-se fazer um teste de reconhecimento de voz contínuo com o binário \texttt{pocketsphinx\_continuous}, criado na compilação do \textit{Pocketsphinx}. Nesta verificação, o usuário fala uma palavra ou uma frase curta, em inglês, em seu microfone. Quando um silêncio é detectado, o programa analisa o \textit{utterance} obtido e imprime na tela o texto que calculou ser a melhor interpretação.

No diretório onde encontram-se as pastas \texttt{sphinxbase/} e \texttt{pocketsphinx/}, execute o conteúdo da listagem \ref{sphinxTest}.

\lstinputlisting[
  language=Bash,
  basicstyle=\scriptsize,
  caption={Comandos para teste de reconhecimento de voz contínuo usando \textit{Pocketsphinx}},
  label={sphinxTest}]
  {listing/run-sphinx-continuous.sh}

O programa imediatamente irá imprimir uma lista de seus parâmetros e seus respectivos valores. Depois, avisará ao usuário que está pronto para receber a entrada de voz por meio de uma linha terminada em \texttt{Ready....}

A listagem \ref{sphinx123} representa uma saída resumida ao se falar \texttt{``one two three''} no microfone. Os caracteres \texttt{[..]} representam uma ou mais linhas omitidas.

\lstinputlisting[
  basicstyle=\scriptsize,
  numbers=left,
  caption={Saída do \texttt{pocketsphinx\_continuous} ao se falar \texttt{``one two three''}},
  label={sphinx123}]
  {listing/sphinx-continuous-123.txt}

Uma interpretação detalhada de toda a saída exige um estudo maior em reconhecimento de voz e na biblioteca \textit{Pocketsphinx} em si. No entanto, destacamos alguma informações, como o número de palavras reconhecidas (linhas 4 e 8) e a quantidade de \textit{senones} (detectores curtos de sons para trifonemas) captadas.

% ---------------------------------------------------------------------

\section{Estruturas e tipos importantes}

Feita a compilação da biblioteca, começaremos a analisar as ferramentas que oferece para implementação de reconhecimento de voz. Ressaltamos que a implementação dos pacotes \textit{Sphinxbase} e \textit{Pocketsphinx} é feita na linguagem \textit{C}.

Dentre os tipos de dados oferecidos pelos pacotes \textit{Sphinxbase} e \textit{Pocketsphinx}, destacamos três, a seguir, que serão importantes para um experimento que faremos em breve.

% ---------------------------------------------------------------------

\subsection{Configuração: \texttt{cmd\_ln\_t}}

A \textit{struct} \texttt{cmd\_ln\_t}, definida no pacote \textit{Sphinxbase}, representa uma variável de configuração. Ela é fornecida a outros tipos de dados em \textit{Pocketsphinx}; informa-se, por exemplo, os arquivos a serem usados (dicionário, modelo acústico, etc.) e se reconhecimento usará um arquivo de áudio ou será feito na hora.

Um ponteiro pode ser alocado e gerenciado com a função \texttt{cmd\_ln\_init()}, devendo-se liberá-lo posteriormente com uma chamada a \texttt{cmd\_ln\_free\_r()}.

Um dos parâmetros existentes neste tipo de configuração é o nome do microfone, no contexto do sistema do usuário. Não conseguimos encontrar uma forma de se obter, em tempo de execução, os nomes dos microfones disponíveis em um sistema \textit{Unix}. Percebemos, também, que seria bastante complicado desenvolver uma solução que funcionasse independentemente do sistema operacional. Portanto, escolhemos passar o nome do microfone como \texttt{NULL}, o que leva \textit{Pocketsphinx} a sempre utilizar o microfone padrão do computador, seja qual for a plataforma em que ele está.

% ---------------------------------------------------------------------

\subsection{Gravação: \texttt{ad\_rec\_t}}

A \textit{struct} \texttt{ad\_rec\_t} está definida no pacote \textit{Sphinxbase} e tem como objetivo gravar som de alguma entrada de voz. Seu uso é vital na implementação de reconhecimento de voz contínuo, através do microfone do usuário.

Aloca-se um ponteiro para este tipo com a função \texttt{ad\_open\_dev()}, cujos parâmetros são um ponteiro para um tipo de configuração \texttt{cmd\_ln\_t} e a taxa de amostragem por segundo. O ponteiro do gravador deve ser liberado posteriormente  com \texttt{ad\_close()}.

As três funções mais importantes para manipulação de um \texttt{ad\_rec\_t} são explicadas abaixo. Todas elas retornam um inteiro diferente de 0 no caso de um erro ocorrer.

\subsubsection{\texttt{int ad\_start\_rec(recorder)}}

Inicia a gravação no seu argumento \texttt{ad\_rec\_t *recorder}.

\subsubsection{\texttt{int ad\_read(recorder, buffer, size)}}

Lê o áudio gravado em \texttt{ad\_rec\_t *recorder} desde a última chamada desta função para este argumento. Guarda-se o áudio lido em um \textit{buffer} do tipo inteiro, que possui o tamanho \texttt{size} especificado.

\subsubsection{\texttt{int ad\_stop\_rec(recorder)}}

Termina a gravação no seu argumento \texttt{ad\_rec\_t *recorder}.

% ---------------------------------------------------------------------

\subsection{Decodificação: \texttt{ps\_decoder\_t}}

A \textit{struct} \texttt{ps\_decoder\_t}, definida no pacote \textit{Pocketsphinx}, representa um decodificador de áudio para texto. Toda a lógica por trás de reconhecimento de voz, portanto, é tratada por funções ligadas a este tipo.

Cria-se um ponteiro para um decodificador com a função \texttt{ps\_init()}, que recebe um tipo de configuração \texttt{cmd\_ln\_t} como seu único argumento. A memória alocada deve ser liberada após seu uso com a função \texttt{ps\_free()}.

Suas funções mais importantes para manipulação são explicadas a seguir. Todas as funções do tipo \texttt{int} retornam um inteiro diferente de 0 no caso de um erro ocorrer.

\subsubsection{\texttt{int ps\_start\_utt(decoder)}}

Inicializa o processamento para \textit{utterance} no \texttt{ps\_decoder\_t *decoder} indicado.

\subsubsection{\texttt{int ps\_process\_raw(decoder, buffer, size, no\_search, full\_utt)}}

Usa-se \texttt{ps\_decoder\_t *decoder} para decodificar o áudio guardado no \textit{buffer} de tamanho especificado. Os dois parâmetros restantes representam pequenas otimizações possíveis: \texttt{no\_search} realiza parcialmente o reconhecimento para usar menos processamento, e \texttt{full\_utt} considera o \textit{buffer} inteiro como um \textit{utterance}.

\subsubsection{\texttt{bool ps\_get\_in\_speech(decoder)}}

Retorna \texttt{true} se o \texttt{ps\_decoder\_t *decoder} fornecido tiver decodificado algo, ou \texttt{false} caso contrário. Esta função só deve ser chamada após um \texttt{ps\_process\_raw()}.

\subsubsection{\texttt{char * ps\_get\_hyp(decoder, \&score)}}

Retorna a hipótese (isto é, o termo mais provável) para um \texttt{ps\_decoder\_t *decoder} que foi processado anteriormente com \texttt{ps\_process\_raw()}. Opcionalmente, se um modelo de língua está sendo usado, pode-se passar a referência de um inteiro como segundo argumento para obter a avaliação (\textit{score}) recebida para a hipótese.

\subsubsection{\texttt{int ps\_stop\_utt(decoder)}}
Termina o processamento para \textit{utterance} no \texttt{ps\_decoder\_t *decoder} indicado.

% ---------------------------------------------------------------------

\section{Implementação de reconhecimento contínuo}

O reconhecimento contínuo de palavras-chave foi implementado com base no código presente em \citep{pocketsphinxMicLoop}. Apresentamos, na listagem \ref{ps-stt-loop}, a função que implementa o laço central do algoritmo, onde repetidamente capta-se a voz do usuário para processamento e imprime-se, na tela, as palavras mais próximas ao que foi pronunciado. Removeram-se as linhas relativas a tratamento de erro para não estender demais a função.

\begin{minipage}{\linewidth}
\lstinputlisting[
  language=C,
  numbers=left,
  caption={Laço de reconhecimento de voz usando-se as ferramentas de \textit{Pocketsphinx}},
  label={ps-stt-loop}]
  {listing/ps-stt-loop.c}
\end{minipage}

As linhas 2 a 9 correspondem à inicialização de variáveis, do gravador e do decodificador. Assume-se, inicialmente, que não estamos no meio de um \textit{utterance} através de \texttt{utt\_started = false} (linha 3).

O laço de reconhecimento de voz começa na linha 11, e usa a seguinte lógica:

\begin{itemize}
\item \textbf{Linhas 12-14}: O áudio de entrada é capturado, e seus dados são processados pelo decodificador.

\item \textbf{Linhas 16-18}: Se foi possível decodificar alguma palavra do usuário, significa que estamos diante de um \textit{utterance}. Em outras palavras, o usuário está falando algo. Coloca-se \texttt{utt\_started = true}.

\item \textbf{Linhas 22-34}: A condição da linha 23 é oposta à que acabamos de ver anteriormente, verificando se o gravador/decodificador não detectaram mais nenhuma palavra vinda do usuário. Caso seja verdade, encerramos o \textit{utterance} atual e adquirimos sua interpretação (hipótese) através de \texttt{ps\_get\_hyp()} (linhas 26 a 28). Por fim, reiniciamos o decodificador para captar um novo \textit{utterance}, alterando \texttt{utt\_started} para \texttt{false} para refletir esta mudança (linhas 31 a 33).
\end{itemize}

O leitor talvez note que a chamada a \texttt{ad\_stop\_rec()} na linha 37 é inalcançável devido ao laço implementado não possuir condição de parada. A implementação foi feita desta forma porque a situação apresentada é um teste; em um caso mais específico, como na implementação do módulo de reconhecimento de voz, a presença de uma condição de parada é vital.
