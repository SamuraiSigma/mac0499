\chapter{Pocketsphinx}
\label{cap:pocketsphinx}

Neste capítulo, analisaremos mais a fundo a biblioteca \textit{Pocketsphinx}, incluindo seu funcionamento, instruções para usá-la de forma básica e passos para compilação a partir do código fonte.

Supõe-se que o usuário esteja usando um sistema operacional \textit{Unix}, e que possua acesso a privilégios administrativos para a realização de alguns passos. Recomenda-se que o leitor possua um microfone à disposição no computador, podendo ser embutido ou externo, para melhor aproveitamento.

Todas as instruções e comandos apresentados foram realizados no sistema \texttt{Ubuntu 16.04 LTS, 64-bit} do autor.

% ---------------------------------------------------------------------

\section{Funcionamento}

% ---------------------------------------------------------------------

\section{Compilação}

Apresentamos instruções, em \textit{Bash}, para baixar e compilar a biblioteca \textit{Pocketsphinx}. Os passos foram baseados nas instruções em \citep{pocketsphinxInstall}.

Antes de começar, instale as seguintes dependências em seu sistema:

\begin{center}
\footnotesize\texttt{gcc, automake, autoconf, libtool, bison, swig, python-dev, pulseaudio}
\end{center}

Em um sistema \emph{Ubuntu}, por exemplo, digitaria-se no terminal:

\begin{lstlisting}[language=Bash]
$ sudo apt-get install gcc automake autoconf libtool bison swig \
  python-dev pulseaudio
\end{lstlisting}

% ---------------------------------------------------------------------

\subsection{Pacote \textit{Sphinxbase}}
\label{sphinxbaseCompile}

O pacote \textbf{Sphinxbase} oferece funcionalidades comuns a todos os projetos \textit{CMUSphinx}. Siga as instruções abaixo para compilá-lo.

\begin{enumerate}
\item Clone o repositório do \textit{Sphinxbase}.

\begin{lstlisting}[language=Bash]
$ git clone https://github.com/cmusphinx/sphinxbase
\end{lstlisting}

\item Dentro do diretório \texttt{sphinxbase/} criado pelo passo anterior, execute o \textit{script} \texttt{autogen.sh} para gerar o arquivo \texttt{configure}:

\begin{lstlisting}[language=Bash]
$ ./autogen.sh
\end{lstlisting}

\item Execute o \textit{script} \texttt{configure} criado no último passo:

\begin{lstlisting}[language=Bash]
# Padrão
$ ./configure

# Plataformas sem aritmética de ponto flutuante
$ ./configure ---enable-fixed ---without-lapack
\end{lstlisting}

Note que qualquer dependência ausente no sistema (por exemplo, o pacote \texttt{swig}) será notificada ao usuário neste passo. Se a execução ocorrer sem problemas, um \texttt{Makefile} será gerado.

\item Compile o \textit{Sphinxbase} através do \texttt{Makefile}:
\begin{lstlisting}[language=Bash]
$ make
\end{lstlisting}
\end{enumerate}

% ---------------------------------------------------------------------

\subsection{Pacote \textit{Pocketsphinx}}
\label{pocketsphinxCompile}

O pacote \textbf{Pocketsphinx} contém as funcionalidades de reconhecimento de voz em si que nos interessam para este trabalho. Siga as instruções abaixo para compilá-lo.

\begin{enumerate}
\item Clone o repositório do \textit{Pocketsphinx}, o que criará o diretório \texttt{pocketsphinx/}.

\begin{lstlisting}[language=Bash]
$ git clone https://github.com/cmusphinx/pocketsphinx
\end{lstlisting}

\item Certifique-se que as pastas \texttt{sphinxbase/} e \texttt{pocketsphinx/} estejam no mesmo diretório, pois \textit{Pocketsphinx} usa o caminho \texttt{../} para procurar pelo pacote \textit{Sphinxbase}.

\item Dentro do diretório \texttt{pocketsphinx/}, execute o \textit{script} \texttt{autogen.sh} para gerar o arquivo \texttt{configure}:

\begin{lstlisting}[language=Bash]
$ ./autogen.sh
\end{lstlisting}

\item Execute o \textit{script} \texttt{configure} criado no último passo:

\begin{lstlisting}[language=Bash]
$ ./configure
\end{lstlisting}

Note que qualquer dependência ausente no sistema será notificada ao usuário neste passo. Se a execução ocorrer sem problemas, um \texttt{Makefile} será gerado.

\item Compile o \textit{Pocketsphinx} através do \texttt{Makefile}:

\begin{lstlisting}
$ make
\end{lstlisting}
\end{enumerate}

% ---------------------------------------------------------------------

\subsection{Teste de verificação}

Para verificar se a compilação feita nas subseções \ref{sphinxbaseCompile} e \ref{pocketsphinxCompile} ocorreu corretamente, recomenda-se fazer um teste de reconhecimento de voz contínuo com o binário \texttt{pocketsphinx\_continuous}, criado na compilação do \textit{Pocketsphinx}. Nesta verificação, o usuário fala uma palavra ou uma frase curta, em inglês, em seu microfone. Quando um silêncio é detectado, o programa analisa o \textit{utterance} obtido e imprime na tela o texto que calculou ser a melhor interpretação.

No diretório onde encontram-se as pastas \texttt{sphinxbase/} e \texttt{pocketsphinx/}, execute o conteúdo da listagem \ref{sphinxTest}:

\begin{lstlisting}[
  language=Bash,
  basicstyle=\scriptsize,
  caption={Comandos para teste de reconhecimento de voz contínuo usando \textit{Pocketsphinx}},
  label={sphinxTest}]
# Diretório contendo arquivos para reconhecimento de voz (modelos, etc.)
MODELDIR=pocketsphinx/model

./pocketsphinx/src/programs/pocketsphinx_continuous \
-inmic yes \                                 # Acionar uso do microfone
-hmm   $MODELDIR/en-us/en-us/mdef \          # Diretório do modelo acústico
-dict  $MODELDIR/en-us/cmudict-en-us.dict \  # Arquivo do dicionário
-lm    $MODELDIR/en-us/en-us.lm.bin          # Arquivo do modelo da língua
\end{lstlisting}

O programa imediatamente irá imprimir uma lista de seus parâmetros e seus respectivos valores. Depois, avisará ao usuário que está pronto para receber a entrada de voz por meio de uma linha terminada em \texttt{Ready....}

A listagem \ref{sphinx123} representa uma saída resumida ao se falar \texttt{``one two three''} no microfone. Os caracteres \texttt{[..]} representam uma ou mais linhas omitidas.

\begin{lstlisting}[
  basicstyle=\scriptsize,
  caption={Saída do \texttt{pocketsphinx\_continuous} ao se falar \texttt{``one two three''}},
  label={sphinx123}]
INFO: continuous.c(275): Ready....
INFO: continuous.c(261): Listening...
[..]
INFO: ngram_search_fwdtree.c(1550):     3081 words recognized (15/fr)
INFO: ngram_search_fwdtree.c(1552):   703838 senones evaluated (3400/fr)
INFO: ngram_search_fwdtree.c(1556):  2241048 channels searched (10826/fr)
[..]
INFO: ngram_search_fwdflat.c(302): Utterance vocabulary contains 154 words
INFO: ngram_search_fwdflat.c(948):     2575 words recognized (12/fr)
INFO: ngram_search_fwdflat.c(950):   148022 senones evaluated (715/fr)
INFO: ngram_search_fwdflat.c(952):   209298 channels searched (1011/fr)
[..]
INFO: ngram_search.c(1381): Lattice has 317 nodes, 942 links
INFO: ps_lattice.c(1380): Bestpath score: -4833
[..]
one two three
\end{lstlisting}
