% ---------------------------------------------------------------------
% Portuguese

\chapter*{Resumo}

A área de \emph{games} evoluiu muito desde o início da década da 70, quando começaram
a ser comercializados. As principais causas estão relacionadas aos avanços em
diferentes áreas da Computação.

Com o passar do tempo, surgiram as \emph{game engines}: \emph{frameworks} voltados
especificamente para a criação de jogos, visando a facilitar o desenvolvimento e/ou
algumas de suas etapas.

Focaremos em uma \emph{game engine} em particular, \emph{Godot}. Por possuir código
aberto, este \emph{software} permite a extensão de suas funcionalidades através da
criação de novos módulos.

Este projeto busca implementar um módulo de reconhecimento de voz para \emph{Godot},
depois demonstrando a nova capacidade em um jogo simples desenvolvido na própria
plataforma.
\\

\noindent \textbf{Palavras-chave:} \emph{software}, \emph{game engine}, \emph{Godot},
desenvolvimento de módulo, extensão de funcionalidade.

% ---------------------------------------------------------------------
% English

\chapter*{Abstract}

Video games have evolved considerably since the beginning of the 70's, when they
started to be commercialized. The main reasons are related to several advances in
different fields of Computer Science.

Over time, \emph{game engines} started appearing: \emph{frameworks} designed
specifically to assist on game creation, simplifying the process and/or some of its
steps.

We will focus on a specific game engine, \emph{Godot}. Since it is an open source
project, it is possible to extend its funcionalities by creating new modules.

This project's goal is to implement a speech recognition module for \emph{Godot},
then showing the new feature in a simple game developed on the engine itself.
\\

\noindent \textbf{Keywords:} software, game engine, \emph{Godot}, module development,
functionality extension.
