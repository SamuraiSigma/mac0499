% ---------------------------------------------------------------------
% Portuguese

\chapter*{Resumo}
\addcontentsline{toc}{chapter}{Resumo}

A área de jogos eletrônicos (\textit{video games}) evoluiu muito desde o início da década da 70, quando começaram a ser comercializados. As principais causas estão relacionadas aos avanços em diferentes áreas da Computação.

Com o passar do tempo, surgiram as \emph{game engines}: \textit{frameworks} voltados especificamente para a criação de jogos, visando a facilitar o desenvolvimento e/ou algumas de suas etapas.

Focaremos em uma \textit{game engine} em particular, \emph{Godot} \citep{godot}. Por possuir código aberto, este \textit{software} permite a extensão de suas funcionalidades através da criação de novos módulos.

Este projeto busca implementar um módulo de reconhecimento de voz para \textit{Godot}, depois demonstrando a nova capacidade em um jogo simples desenvolvido na própria plataforma.
\par
\bigskip
\noindent \textbf{Palavras-chave:} \textit{software}, \textit{game engine}, \textit{Godot}, desenvolvimento de módulo, extensão de funcionalidade, reconhecimento de voz.

% ---------------------------------------------------------------------
% English

\chapter*{Abstract}
\addcontentsline{toc}{chapter}{Abstract}

Video games have evolved considerably since the beginning of the 70's, when they started to be commercialized. The main reasons are related to several advances in different fields of Computer Science.

Over time, \emph{game engines} started appearing: \textit{frameworks} designed specifically to assist on game creation, simplifying the process and/or some of its steps.

We will focus on a specific game engine, \emph{Godot} \citep{godot}. Since it is an open source project, it is possible to extend its funcionalities by creating new modules.

This project's goal is to implement a speech recognition module for \textit{Godot}, then showing the new feature in a simple game developed on the engine itself.
\par
\bigskip
\noindent \textbf{Keywords:} software, game engine, \textit{Godot}, module development, functionality extension, speech recognition.
