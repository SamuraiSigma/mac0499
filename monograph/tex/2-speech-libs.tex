\chapter{Biblioteca para Reconhecimento de Voz}
\label{cap:speech-libs}

% ---------------------------------------------------------------------

\section{Considerações iniciais}

Se o objetivo do projeto é criar um módulo de reconhecimento de voz, primeiramente é
necessário chegar a uma biblioteca que implemente tal funcionalidade. Uma
implementação do zero fugiria do tema deste trabalho, pois seria necessário aprender
sobre reconhecimento de padrões voltado a sons, assunto relacionado a Inteligência
Artificial.

A outra opção existente, e a que seguiremos, é procurar por uma biblioteca existente
e aprender a manejá-la. Idealmente, gostaríamos que a biblioteca possuísse as
seguintes características:

\begin{enumerate}
\item \emph{Ter código aberto e licença permissiva:} Provavelmente o único aspecto
essencial, uma vez que integraremos a biblioteca em uma \emph{game engine}. A
importância é ainda maior se levarmos em conta que jogos com fins comerciais podem
ser produzidos em \emph{Godot}.

% TODO: Fix chapter 3 reference later
\item \emph{Ser implementada em C/C++:} Toda a base do \emph{Godot} é construída em
C++. A criação de módulos utiliza essa mesma linguagem, como será visto no capítulo
3.

\item \emph{Reconhecer inglês:} Escolheremos inglês como uma linguagem obrigatória
a se ter na biblioteca pelo caráter universal dessa linguagem.

\item \emph{Ser multiplataforma:} Ao menos, oferecer suporte para sistemas Windows,
MacOS e Linux.

\item \emph{Implementar eficientemente reconhecimento de voz:} A biblioteca será
usada em jogos, onde eficiência (velocidade) é uma questão central.

\item \emph{Não ser pesada:} Embora tenha menos importância que os aspectos acima,
não é desejável ter uma biblioteca que ocupe muito espaço.
\end{enumerate}

Após uma pesquisa, que pode ser resumida pelo artigo em \citep{speechRecog:16},
cinco bibliotecas de reconhecimento de voz em geral se destacam por seu uso:

\begin{itemize}
\item \emph{Kaldi} \citep{kaldi}: É a biblioteca mais recente da lista, com seu
código publicado em 2011. Escrita em C++.

\item \emph{CMUSphinx} \citep{cmusphinx}: Desenvolvida pela \emph{Carnegie Mellon
University}, possui diversos pacotes para diferentes tarefas e aplicações. O pacote
principal é escrito em Java. Existe também a variante \emph{Pocketsphinx}, com
características interessantes para este trabalho: é escrita em C, possuindo maior
velocidade e portabilidade que a biblioteca original.

\item \emph{HTK} \citep{htk}: Desenvolvida pela \emph{Cambridge University
Engineering Department}, HTK é uma sigla para \emph{Hidden Markov Model Toolkit}. É
escrita em C, com novas versões sendo lançadas consistentemente.

\item \emph{Simon} \citep{Simon}: Popular para Linux e escrita em C++, Simon utiliza
\emph{CMUSphinx}, \emph{HTK} e \emph{Julius} internamente. Não havia suporte para
\emph{MacOS} até 3 de abril de 2017.

\item \emph{Julius} \citep{Julius}: Desenvolvida pela \emph{Interactive Speech
Technology Consortium} e escrita em C. Infelizmente, o suporte para inglês é
limitado e não pode ser usado para propósitos comerciais, o que nos força a
descartar esta biblioteca como uma possível opção.
\end{itemize}

Das cinco bibliotecas descritas, as mais viáveis são as três primeiras (\emph{Kaldi},
\emph{CMUSphinx} e \emph{HTK}). \emph{Simon} está em seus primeiros passos para
MacOS, por isso a relutância em seu uso.

Um artigo de 2014 comparou \emph{Kaldi}, \emph{CMUSphinx} e \emph{HTK} em relação a
precisão e tempo gasto \citep{compareSpeech}. \emph{Kaldi} obteve resultados
vastamente superiores; \emph{CMUSphinx} obteve bons resultados em pouco tempo;
\emph{HTK} precisou de muito mais tempo e treino para conseguir resultados na ordem
dos outros dois.

No restante deste capítulo, analisaremos com mais cuidado as bibliotecas
\emph{Kaldi}, \emph{PocketSphinx} e \emph{HTK}, com o intuito de acompanhar mais de
perto suas funcionalidades, vantagens e desvantagens.

O sistema operacional utilizado foi \texttt{ubuntu 16.04 LTS, 64 bits}.

% ---------------------------------------------------------------------

\section{\emph{Pocketsphinx}}

\subsection{Instalação}

Os passos abaixo foram baseados nas instruções em \citep{pocketsphinxInstall}.
Supõe-se que o usuário esteja usando um sistema \emph{Unix}, com acesso a privilégios
administrativos (\emph{root}).

Antes de começar, é necessário instalar as seguintes dependências: \texttt{gcc,
automake, autoconf, libtool, bison, swig, python-dev, pulseaudio}. Se estiver usando
um sistema como \emph{Ubuntu}, por exemplo, digite no terminal:
\shellcmd{sudo apt-get install gcc automake autoconf libtool bison \textbackslash
\linebreak \hphantom{4} swig python-dev pulseaudio}

Primeiramente, baixaremos e instalaremos o pacote \emph{Sphinxbase}, que oferece
funcionalidades comuns a todos os projetos \emph{CMUSphinx}:

\begin{enumerate}
\item Clone o repositório do \emph{Sphinxbase}, o que resultará no diretório
\texttt{sphinxbase}:
\shellcmd{git clone https://github.com/cmusphinx/sphinxbase}

\item Dentro do diretório \texttt{sphinxbase}, execute o \emph{script}
\texttt{autogen.sh} para gerar o arquivo \texttt{configure}:
\shellcmd{./autogen.sh}

\item Execute o \emph{script} \texttt{configure} que foi criado no último passo:
\shellcmd{./configure}

Se a plataforma utilizada não possui aritmética de ponto flutuante, deve-se rodar ao
invés disso:
\shellcmd{./configure ---enable-fixed ---without-lapack}

Note que qualquer dependência ausente no sistema (por exemplo, o pacote \emph{swig})
será notificada ao usuário neste passo. Se a execução ocorrer sem problemas, um
\texttt{Makefile} será gerado.

\item Construa a biblioteca através do \texttt{Makefile}:
\shellcmd{make}

\item Instale a biblioteca \emph{Sphinxbase} no sistema:
\shellcmd{sudo make install}

\emph{Observação:} Para desinstalar a biblioteca \emph{Sphinxbase} do sistema, basta digitar:
\shellcmd{sudo make uninstall}
\end{enumerate}

Os passos necessários para baixar e instalar a biblioteca \emph{Pocketsphinx} são
semelhantes:

\begin{enumerate}
\item Clone o repositório do \emph{Pocketsphinx}, o que resultará no diretório
\texttt{pocketsphinx}:
\shellcmd{git clone https://github.com/cmusphinx/pocketsphinx}

\item Certifique-se que os diretórios \texttt{sphinxbase} e \texttt{pocketsphinx}
estejam no mesmo diretório.

\item Dentro do diretório \texttt{pocketsphinx}, execute o \emph{script}
\texttt{autogen.sh} para gerar o arquivo \texttt{configure}:
\shellcmd{./autogen.sh}

\item Execute o \emph{script} \texttt{configure} que foi criado no último passo:
\shellcmd{./configure}

Note que qualquer dependência ausente no sistema será notificada ao usuário neste
passo. Se a execução ocorrer sem problemas, um \texttt{Makefile} será gerado.

\item Construa a biblioteca através do \texttt{Makefile}:
\shellcmd{make}

\item Instale a biblioteca \emph{Pocketsphinx} no sistema:
\shellcmd{sudo make install}

\emph{Observação:} Para desinstalar a biblioteca \emph{Pocketsphinx} do sistema,
basta digitar:
\shellcmd{sudo make uninstall}
\end{enumerate}
