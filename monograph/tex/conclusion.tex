\chapter{Conclusão}
\label{cap:conclusion}

Este Trabalho de Conclusão de Curso envolveu o desenvolvimento de um módulo de reconhecimento de voz para a \textit{game engine} \textit{Godot}. Para tanto, dedicou-se um tempo razoável do trabalho para o estudo das principais características de sistemas de reconhecimento de voz. Ficou evidente o vasto uso que esta tecnologia possui hoje, e o futuro indica que seu uso poderá crescer ainda mais. A tendência de ideologias como \emph{Internet das Coisas} é trazer cada vez mais poder computacional para automatizar tarefas repetitivas na vida do ser humano, e interação por voz se encaixaria bem para tais situações.

A procura por bibliotecas de código aberto que atendam a certas expectativas veio a seguir, levando-nos a estudar \textit{Pocketsphinx} como a opção mais viável no contexto de usabilidade dentro de um jogo. Apesar da documentação da biblioteca ser por vezes incompleta, vários exemplos em fóruns possibilitaram um aprendizado satisfatório de seu uso.

\textit{Godot} possui uma documentação bastante completa quanto a seu uso para jogos. No entanto, não há muito material disponível relacionado a suas classes em \textit{C++}. A leitura e entendimento de código, portanto, foi uma habilidade bastante exercida nesta parte, uma vez que certas implementações no módulo \textit{Speech to Text} foram baseadas em algo já feito em outra classe.

A produção do jogo demonstrativo \textit{Color Clutter} teve importância considerável no final deste trabalho, pois mostrou que o módulo correspondeu às expectativas referentes a eficiência e simplicidade de uso.

Em geral, o resultado final foi bastante satisfatório, uma vez que o módulo \textit{Speech to Text} foi até divulgado, com seu código aberto, em fóruns da \textit{game engine}, junto ao jogo \textit{Color Clutter} produzido. Deseja-se continuar o suporte ao módulo no futuro, possivelmente estudando-se mais sobre \textit{Android} e \textit{MacOS} para tentar portá-lo a estes sistemas operacionais.
