\chapter{Bibliotecas para Reconhecimento de Voz}
\label{cap:speech-libs}

A seguir, veremos o primeiro item necessário para atingirmos o objetivo final: uma biblioteca que fará o reconhecimento de voz dentro do módulo.

Uma implementação do zero fugiria do tema deste trabalho, pois seria necessário aprender sobre reconhecimento de padrões voltado a sons e outros tópicos relacionados a Inteligência Artificial. A outra opção existente, e a que seguiremos, é procurar por uma biblioteca existente e aprender a manejá-la.

Analisaremos quais as características necessárias e desejáveis na biblioteca ideal, e estudaremos a que melhor se adequa ao nosso objetivo dentre as opções existentes.

% ---------------------------------------------------------------------

\section{Considerações iniciais}

Recordemos os principais componentes para reconhecimento de voz, apresentados na seção \ref{sttComponents}. No contexto do módulo de reconhecimento de voz para \textit{Godot}, as seguintes associações surgem naturalmente:

\begin{itemize}
\item O \textbf{usuário} representa tipicamente o \textbf{jogador}, que interage parcialmente ou totalmente com o jogo por meio de comandos de voz.

\item O \textbf{dispositivo de STT} corresponde ao \textbf{módulo de reconhecimento de voz}, objetivo principal deste trabalho. Esta componente é usada pelo jogo para converter a fala do jogador em texto.

\item O \textbf{software de aplicação} é o \textbf{jogo} em si, feito em \textit{Godot}, que recebe indiretamente os comandos do usuário e realiza ações apropriadas.
\end{itemize}

% ---------------------------------------------------------------------

\section{A biblioteca ideal}
\label{idealLibrary}

Realçamos novamente que o módulo de reconhecimento de voz será usado diretamente em jogos. Tal contexto automaticamente nos leva a pensar em diversas características que a biblioteca ideal deve possuir.

% ---------------------------------------------------------------------

\subsection{Características obrigatórias}

Em ordem decrescente de importância, temos:

\begin{enumerate}
\item \textbf{Ter código aberto e licença permissiva:} Justifica-se pela integração da biblioteca em uma \textit{game engine} de código aberto. A importância é ainda maior se levarmos em conta que jogos com fins comerciais podem ser produzidos em \textit{Godot}.

\item \textbf{Ser eficiente (rápida):} Já foi mencionado que o módulo de reconhecimento de voz será usado em uma \textit{game engine}. Um jogo é um \textit{software} onde tipicamente a eficiência é de extrema importância, pois costuma envolver a renderização de cenas várias vezes por segundo. Devido a isso, surge a necessidade da biblioteca ser \emph{rápida} para não afetar negativamente a experiência do jogador.

\item \textbf{Reconhecer inglês:} O inglês possui presença constante em cenários de computação. Portanto, é a única língua que a biblioteca deve obrigatoriamente oferecer suporte.

\item \textbf{Não ser pesada:} Não é desejável ter uma biblioteca que ocupe muito espaço em disco (o que poderia aumentar o tamanho do jogo que a utiliza) e memória (aspecto relacionado diretamente à eficiência).
\end{enumerate}

% ---------------------------------------------------------------------

\subsection{Características desejáveis}

Em ordem decrescente de importância, temos:

\begin{enumerate}
\item \textbf{Ser multiplataforma:} \textit{Godot} possibilita exportar jogos para diferentes plataformas, dentre elas Windows, MacOS, Unix, Android e iOS \citep{godotDeployPlatforms}. Uma biblioteca que possa ser compatível com o maior número possível destes sistemas operacionais tornaria o módulo de reconhecimento de voz mais flexível para a produção de jogos em diferentes ambientes.

\item \textbf{Reconhecer diferentes línguas:} Apesar da obrigatoriedade do inglês, a possibilidade de usar diferentes línguas aumentaria a versatilidade do módulo. Tal característica é acentuada ao notarmos que muitos jogos, hoje em dia, oferecem a possibilidade de alterar a língua.

\item \textbf{Ser implementada em C/C++:} Conforme veremos na seção \ref{cap:godot}, \textit{Godot} possui toda a sua base escrita em C++, linguagem também usada para a criação de módulos. A implementação da biblioteca na mesma linguagem ajudaria a simplificar problemas de compatibilidade. Eventualmente, C também é uma opção viável por ser aceita pela linguagem sucessora.
\end{enumerate}

% ---------------------------------------------------------------------

\section{Bibliotecas viáveis}

Realizou-se uma pesquisa por bibliotecas de reconhecimento de voz que sigam o máximo de características possíveis propostas na seção \ref{idealLibrary}. O artigo \citep{sttLibs} sintetiza razoavelmente bem os resultados da busca. A seguir, destacamos as quatro bibliotecas mais notáveis encontradas:

\begin{itemize}
\item \textbf{Kaldi} \citep{kaldi}: É a biblioteca mais recente da lista, com seu código publicado em 2011. Escrita em C++, é tida como uma biblioteca para pesquisadores de reconhecimento de voz.

\item \textbf{CMUSphinx} \citep{cmusphinx}: Desenvolvida pela \textit{Carnegie Mellon University}, possui diversos pacotes para diferentes tarefas e aplicações. O pacote principal é escrito em Java. Existe também a variante \emph{Pocketsphinx}, com características interessantes para este trabalho: é escrita em C, possuindo maior velocidade e portabilidade que a biblioteca original.

\item \textbf{HTK} \citep{htk}: Desenvolvida pela \textit{Cambridge University Engineering Department}, HTK é uma sigla para \textit{Hidden Markov Model Toolkit}. É escrita em C, com novas versões sendo lançadas consistentemente.

\item \textbf{Simon} \citep{Simon}: Popular para Linux e escrita em C++, Simon utiliza \textit{CMUSphinx}, \textit{HTK} e \textit{Julius} internamente. Não havia suporte para \textit{MacOS} até abril de 2017.
\end{itemize}

% TODO: Explain better each library and the comparison article
Um artigo de 2014 comparou \emph{Kaldi}, \emph{CMUSphinx} e \emph{HTK} em relação a precisão e tempo gasto \citep{compareSpeech}. \emph{Kaldi} obteve resultados vastamente superiores; \emph{CMUSphinx} obteve bons resultados em pouco tempo; \emph{HTK} precisou de muito mais tempo e treino para conseguir resultados na ordem dos outros dois.

% ---------------------------------------------------------------------

\section{\textit{Pocketsphinx}, a biblioteca escolhida}

% TODO: Write this section
