\chapter{Módulo \textit{Speech to Text} para \textit{Godot}}
\label{cap:stt-module}

Após adquirirmos conhecimento sobre a biblioteca \textit{Pocketsphinx} e \textit{Godot}, enfim chegou o momento de construirmos o módulo de reconhecimento de voz.

Este capítulo documenta as instruções necessárias para adicionar novas funcionalidades a \textit{Godot}, bem como as decisões de projeto tomadas na criação do módulo, a qual chamaremos de \textit{Speech to Text}.

Pressupomos que o leitor esteja familiarizado com as instruções para compilação de \textit{Godot}, vistas na seção \ref{godotCompile}.

% ---------------------------------------------------------------------

\section{Módulos em \textit{Godot}}

Conforme descrito na seção \ref{godotLanguages}, a linguagem \textit{GDScript} é extremamente prática para programar estruturas em um jogo feito em \textit{Godot}. No entanto, às vezes deseja-se otimizar alguma parte crítica através de \textit{C++} ou adicionar uma nova funcionalidade inexistente em \textit{Godot}. Os módulos servem justamente para este objetivo, pois não fazem parte do código essencial da \textit{game engine}.

Todos os módulos ativos são encontrados como subdiretórios dentro da pasta \texttt{modules/} no código fonte. A seguir, forneceremos as instruções necessárias para a criação de um módulo genérico.

% ---------------------------------------------------------------------

\subsection{Instruções para criação}

Como exemplo, mostraremos como criar o módulo \textit{sumator} no decorrer das instruções.

\begin{enumerate}
\item Crie um diretório, dentro da pasta \texttt{modules/} no código fonte, com o nome do módulo. Neste caso, seria o diretório \texttt{sumator/}.

\begin{lstlisting}
$ cd modules
$ mkdir sumator
\end{lstlisting}

\item Dentro do diretório criado para o módulo, escreva o código \textit{C++} para quaisquer interfaces (arquivos \texttt{.h}) a serem usadas. Como nosso exemplo \textit{Sumator} é bastante simples, criaremos apenas um arquivo: \texttt{sumator.h}. Seu conteúdo é apresentado na listagem \ref{sumator-h}.

\lstinputlisting[
language=C++,
numbers=left,
caption={Arquivo de interface \texttt{sumator.h} para o módulo \textit{Sumator}},
label={sumator-h}]
{listing/sumator.h}

Na linha 6 da listagem \ref{sumator-h}, fizemos a classe \textit{Sumator} herdar de \textit{Node} (visto na seção \ref{godotNode}) para que o módulo possa ser usado de forma direta no editor \textit{Godot}.

Toda classe que herda de alguma classe essencial da \textit{game engine} exige uma chamada a \texttt{OBJ\_TYPE()} (linha 7), que recebe como argumentos o nome da classe e de quem ela herda.

Por fim, vamos nos atentar a \texttt{\_bind\_methods()}, definido na linha 11: este método define quais métodos da classe poderão ser usados em \textit{GDScript}. Seu nome é uma convenção, definido por \textit{Godot} para facilmente localizar o método em uma classe.

\item Escreva o código \textit{C++} para quaisquer implementações (arquivos \textit{.cpp}) necessárias. O exemplo do \textit{Sumator} exige um arquivo \texttt{sumator.cpp} para implementar os métodos definidos no passo anterior. Apresentamos seu conteúdo na listagem \ref{sumator-cpp}.

\lstinputlisting[
language=C++,
numbers=left,
caption={Arquivo de implementação \texttt{sumator.cpp} para o módulo \textit{Sumator}},
label={sumator-cpp}]
{listing/sumator.cpp}

A implementação de \texttt{\_bind\_methods()} tipicamente envolve chamadas ao \textit{singleton} \textit{ObjectTypeDB}, que contém um banco de dados para todas as classes e funcionalidades utilizáveis em \textit{GDScript}.

Nas linhas 16 a 18 da listagem \ref{sumator-cpp}, usamos \texttt{bind\_method()} para amarrar nomes (1º argumento) à referência de métodos (2º argumento), permitindo seu uso em \textit{GDScript}.
\end{enumerate}

% ---------------------------------------------------------------------

\section{Organização de \textit{Speech to Text}}

% ---------------------------------------------------------------------

\section{Divulgação}
