\chapter{Módulo \textit{Speech to Text} para \textit{Godot}}
\label{cap:stt-module}

Após adquirirmos conhecimento sobre a biblioteca \textit{Pocketsphinx} e a \textit{game engine} \textit{Godot}, chegou o momento de construirmos o módulo de reconhecimento de voz.

Conforme descrito na seção \ref{godotLanguages}, a linguagem \mbox{\textit{GDScript}} é extremamente prática para programar estruturas em um jogo feito em \textit{Godot}. No entanto, às vezes deseja-se otimizar alguma parte crítica através de \textit{C++} ou adicionar uma nova funcionalidade inexistente em \textit{Godot}. Os módulos servem justamente para este objetivo, pois não fazem parte do código essencial da \textit{game engine}.

Este capítulo documenta os passos e decisões de projeto tomados na criação do módulo, a qual chamaremos de \textit{Speech to Text}. Pressupomos que o leitor esteja familiarizado com as instruções para compilação de \textit{Godot}, vistas na seção \ref{godotCompile}, e com a biblioteca \textit{Pocketsphinx} (capítulo \ref{cap:pocketsphinx}), que será usada para a realização do reconhecimento de voz.

% ---------------------------------------------------------------------

\section{Primeiros passos}

Antes de começarmos a implementar o módulo em si, precisaremos tomar algumas medidas simples de preparação.

\subsection{Criação do diretório do módulo}

Todos os módulos ativos são encontrados como subdiretórios dentro da pasta \texttt{modules/} no código fonte. Começaremos, portanto, com a criação do diretório \texttt{speech\_to\_text}:

\begin{lstlisting}[language=Bash]
$ cd modules
$ mkdir speech_to_text
$ cd speech_to_text
\end{lstlisting}

% ---------------------------------------------------------------------

\subsection{Adição do pacote \textit{Sphinxbase}}

Usaremos a biblioteca \textit{Pocketsphinx} para realizar o reconhecimento de voz no módulo. Um dos requisitos necessários para seu funcionamento, conforme visto na seção \ref{sphinxCompile}, é o pacote \textit{Sphinxbase}. A seguir, apresentamos instruções para inserir os arquivos essenciais deste pacote no diretório do módulo.

\begin{enumerate}
\item Baixe o pacote \textit{Sphinxbase} em sua versão atual mais estável, a \textbf{5-prealpha}.

\begin{lstlisting}
$ SPHINXURL="https://sourceforge.net/projects/cmusphinx/files"
$ wget $SPHINXURL/sphinxbase/5prealpha/sphinxbase-5prealpha.tar.gz
\end{lstlisting}

\item Extraia e renomeie o pacote baixado.

\begin{lstlisting}
$ tar -xvf sphinxbase-5prealpha
$ mv sphinxbase-5prealpha sphinxbase
\end{lstlisting}

\item Remova arquivos supérfluos, como \texttt{Makefiles}, arquivos de teste e \textit{scripts} de compilação. Estes serviriam para aumentar, desnecessariamente, o tamanho do módulo. Em outras palavras, somente as interfaces e implementações nos pacotes serão mantidas. Veja a listagem \ref{sphinxbaseEssential} para maiores detalhes.

\lstinputlisting[
  language=Bash,
  caption={Remoção de arquivos supérfluos no pacote \textit{Sphinxbase}},
  label={sphinxbaseEssential}]
  {listing/sphinxbase-essential.sh}
\end{enumerate}

% ---------------------------------------------------------------------

\subsection{Adição do pacote \textit{Pocketsphinx}}

Precisamos, também, do pacote \textit{Pocketsphinx} para a biblioteca homônima funcionar. A seguir, apresentamos instruções para inserir os arquivos essenciais deste pacote no diretório do módulo.

\begin{enumerate}
\item Baixe o pacote \textit{Pocketsphinx} em sua versão atual mais estável, a \textbf{5-prealpha}.

\begin{lstlisting}
$ SPHINXURL="https://sourceforge.net/projects/cmusphinx/files"
$ wget $SPHINXURL/pocketsphinx/5prealpha/pocketsphinx-5prealpha.tar.gz
\end{lstlisting}

\item Extraia e renomeie o pacote baixado.

\begin{lstlisting}
$ tar -xvf pocketsphinx-5prealpha
$ mv pocketsphinx-5prealpha pocketsphinx
\end{lstlisting}

\item Remova arquivos supérfluos, isto é, que não sejam interfaces ou implementações. Veja a listagem \ref{pocketsphinxEssential} para maiores detalhes.

\lstinputlisting[
  language=Bash,
  caption={Remoção de arquivos supérfluos no pacote \textit{Pocketsphinx}},
  label={pocketsphinxEssential}]
  {listing/pocketsphinx-essential.sh}
\end{enumerate}

% ---------------------------------------------------------------------

\section{Planejamento}

Quais os requisitos funcionais e não funcionais que queremos atender? O que um típico usuário do módulo \textit{Speech to Text} desejaria para poder usar em seu jogo? Todo projeto deve começar com algum planejamento mínimo de onde se quer chegar para obter algum sucesso.

% ---------------------------------------------------------------------

\subsection{Requisitos não funcionais}

% TODO: Fix references

Reconhecimento de voz em jogos geralmente é usado em um contexto de tempo real. Isto é, para uma dada fala do usuário, não desejamos que o jogo demore muito para dar alguma forma de resposta com o risco de comprometer seu aspecto lúdico. A preocupação é reduzida ao lembrarmos que \textit{Pocketsphinx} apresentou resultados rápidos nos testes realizados na seção \ref{???}.

Portanto, em termos dos principais parâmetros de reconhecimento de voz definidos na seção \ref{sttMainTerms}, projetaremos o módulo com a questão de \textbf{eficiência} em mente:

\begin{itemize}
\item \textbf{Fluência}: Idealmente, a forma de comunicação poderia chegar até \emph{palavras conectadas}. Uma \textit{fala contínua} demandaria um processamento muito pesado e comprometedor para o jogo.

\item \textbf{Dependência do usuário}: Um sistema \emph{independente} é mais flexível por atender a uma maior quantidade de pessoas sem a necessidade de um longo treinamento antes de começaram a jogar.

\item \textbf{Vocabulário}: Deve ser tipicamente \emph{pequeno} (não mais do que 40 palavras). Um vocabulário muito grande aumentaria a velocidade de reconhecimento, que por sua vez afetaria a experiência do jogador.

\item \textbf{Parâmetros ambientais}: Não é esperado que interfiram tanto no jogo. A relação sinal/ruído deve ser baixa, pois um ambiente muito barulhento comprometeria a jogabilidade. Por fim, desejamos que o usuário possa falar em um tom de voz normal, sem precisar ``forçar'' a pronúncia das palavras ou aumentar sem tom para o reconhecimento ser possível. Tais características são automaticamente tratadas pelo modelo acústico usado no \textit{Pocketsphinx}.
\end{itemize}

Desejamos que o módulo seja \textbf{confiável} em relação à acurácia do reconhecimento de voz. Conforme vimos na seção \ref{???}, isto dependerá dos modelos usados na configuração do \textit{Pocketsphinx}.

\textbf{Configurabilidade} também é uma característica desejada: o usuário do módulo deverá ter controle sobre a língua do reconhecimento e o vocabulário reconhecido, por exemplo. \textit{Pocketsphinx} permite isso facilmente com a alteração de seus arquivos de configuração, incluindo o modelo acústico e as palavras-chave.

O módulo deverá ser de \textbf{propósito geral} em relação ao tipo de jogo em que é empregado (ação, terror, plataforma, etc.); portanto, não é possível prever características do típico usuário do jogo. Este requisito, em geral, não é tão preocupante quando levamos em conta o uso de um modelo acústico geral com a biblioteca \textit{Pocketsphinx}.

É importante que o módulo seja \textbf{tolerante a erros}, isto é, que comunique ao restante do sistema quando um problema ocorre dentro de si.

Por fim, um requisito desejável, mas a princípio não estritamente necessário, é a  \textbf{portabilidade}. Apesar das classes internas de \textit{Godot} e a biblioteca \textit{Pocketsphinx} terem sido projetados para funcionarem em diversos sistemas operacionais, colocaremos plataformas \textit{Unix} como a meta principal. Suporte a outras plataformas poderá ser feito depois, dependendo da complexidade da implementação.

% ---------------------------------------------------------------------

\subsection{Requisitos funcionais}

A princípio, desejamos que o reconhecimento de voz seja executado em paralelo com o restante do jogo. Isto é, gostaríamos que a execução não parasse totalmente até obter um comando por voz do usuário. Queremos, também, uma forma de verificar se o reconhecimento está ativo e de iniciá-lo/desligá-lo a qualquer momento.

As palavras reconhecidas pelo módulo \textit{Speech to Text} não precisariam ser interpretadas imediatamente pelo jogo. Uma ideia mais flexível é guardá-las em um \textit{buffer} e deixar o próprio jogo lê-las em seu ritmo.

% TODO: Update reference on Pocketsphinx keywords section below

Como o vocabulário deve ser tipicamente pequeno, o reconhecimento por \emph{palavras-chaves} do \textit{Pocketsphinx} é bem mais viável do que por modelo de língua.

A configurabilidade desejada nos requisitos não funcionais nos leva a precisar de uma interface para ajustar parâmetros e arquivos do reconhecimento de voz. Em particular, um usuário estaria interessado em configurar o modelo acústico, o dicionário e as palavras-chave.

% ---------------------------------------------------------------------

\section{Implementação}

Apresentamos, na figura \ref{sttModuleClassDiagram}, um diagrama de classes do módulo \textit{Speech to Text}.

\begin{figure}[H]
  \centering
  \includegraphics[width=\textwidth]{image/stt-module.pdf}
  \caption{Diagrama de classes do módulo \textit{Speech to text}}
  \label{sttModuleClassDiagram}
\end{figure}

Omitimos alguns métodos privados e os construtores/destrutores para simplificar o diagrama. \textit{Godot}, inclusive, não permite que construtores e destrutores possuam argumentos como uma forma de padronizar o instanciamento e liberação de objetos.

Os tipos \texttt{String} e \texttt{Vector} usados não são os da STL (\textit{Standard Template Library}) de \textit{C++}, mas sim implementações próprias de \textit{Godot} para essas estruturas.

Quase todas as classes implementadas, exceto \textit{FileDirUtil}, possuem um método \texttt{\_bind\_methods()}. Este método, padronizado em classes da \textit{game engine}, é usado para ligar nomes a constantes ou referências de métodos, permitindo seu uso em \mbox{\textit{GDScript}}; todas as ligações são guardadas numa grande classe \textit{singleton} denominada \textit{ObjectTypeDB}. A listagem \ref{configBindMethods} exemplifica a adição de uma linha no método \texttt{\_bind\_methods()} de \textit{STTConfig} para ser possível usar o método \texttt{init()} desta mesma classe em \mbox{\textit{GDScript}}.

\begin{lstlisting}[
  language=C++,
  label=configBindMethods,
  caption={Adicionando o método \texttt{init()} de \textit{STTConfig} para uso em \mbox{\textit{GDScript}}}
]
void STTConfig::_bind_methods() {
    ObjectTypeDB::bind_method("init", &STTConfig::init);
}
\end{lstlisting}

A seguir, detalharemos cada classe presente na figura \ref{sttModuleClassDiagram}.

% ---------------------------------------------------------------------

\subsection{Classe \textit{STTConfig}}

% ---------------------------------------------------------------------

\subsection{Classe \textit{STTRunner}}

% ---------------------------------------------------------------------

\subsection{Classe \textit{STTQueue}}

% ---------------------------------------------------------------------

\subsection{Classe \textit{STTError}}

% ---------------------------------------------------------------------

\subsection{Classe \textit{FileDirUtil}}

% ---------------------------------------------------------------------

\section{Divulgação}

Todo o código fonte do módulo \textit{Speech to Text} encontra-se no GitHub do autor \citep{sttModuleGitHub}. O repositório também disponibiliza binários dos editores \textit{Godot} e \textit{templates de exportação} já contendo o módulo \citep{sttModuleDownload}.

\textit{Speech to Text} foi divulgado em dois fóruns de \textit{Godot}, onde obteve algumas poucas aprovações:

\begin{itemize}
\item \textbf{Godot Engine Q\&A} \citep{sttModuleGodotQA}: O site oficial de \textit{Godot} disponibiliza uma seção para perguntas e respostas, onde existe uma subseção para publicação de projetos.

\item \textbf{Godot Developers} \citep{sttModuleGodotDevelopers}: Embora seja mais voltado para jogos produzidos na \textit{game engine}, este fórum possui uma seção para compartilhamento de recursos e ferramentas.
\end{itemize}
