\chapter*{Disciplinas importantes para este trabalho}
\addcontentsline{toc}{chapter}{Disciplinas importantes para este trabalho}

Dentre as disciplinas que cursei durante a graduação, as seguintes se destacam pela relação maior com a confecção deste trabalho:

\subsubsection{(MAC0110) Introdução a Computação \\
(MAC0122) Princípios de Desenvolvimento de Algoritmos \\
(MAC0323) Estrutura de Dados}

Entrei no Bacharelado em Ciência da Computação sem conhecimento algum de programação, mas com o intuito de aprender bastante sobre o assunto. Estas três disciplinas fornecerem a base que eu desejava, e mostraram que programar não é apenas redigir código. Escrever um programa eficiente, robusto e correto exige raciocínio lógico, capacidade de abstração e domínio das estruturas de dados disponíveis.

\subsubsection{(MAC0211) Laboratório de Programação I \\
(MAC0242) Laboratório de Programação II}

Estas disciplinas possuem um foco mais prático de ferramentas e concepções para programação em si. Destacam-se o uso de linguagens de \textit{script}, expressões regulares e Programação Orientada a Objetos. Meu primeiro contato com UML também ocorreu nestas disciplinas.

\subsubsection{(MAC0332) Engenharia de Software}

Apresentou metodologias interessantes para a produção de \textit{software}. Também tive um forte contato com diagramas UML. Sinto que a classificação de aplicações de acordo com sua finalidade e técnicas importantes para planejamento me ajudaram, de certa forma, na preparação dos requisitos do módulo de reconhecimento de voz.

\subsubsection{(MAC0441) Programação Orientada a Objetos}

Minha primeira experiência com a produção de um \textit{software} de proporção bem maior que um típico Exercício-Programa, pois fora realizado um projeto sobre Cidades Inteligentes que englobou a classe inteira. Ajudou a aprofundar meus conhecimentos sobre Orientação a Objetos e a produção de código de qualidade.

\subsubsection{(MAC0422) Sistemas Operacionais}

Apresentou a ideia de \textit{threads} pela primeira vez no curso, além de mostrar as componentes principais de sistemas operacionais. Em especial, destacamos o gerenciamento de arquivos, que apareceu, neste trabalho, na implementação própria do sistema de arquivos de \textit{Godot}.

\subsubsection{(MAC0425) Inteligência Artificial}

Alguns tópicos, como \textit{Markov Decision Process} (MDP), lembram o procedimento estocástico comentado em \textit{Hidden Markov Model} (HMM).
