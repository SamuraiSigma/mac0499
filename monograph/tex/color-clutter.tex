\chapter{Jogo \textit{Color Clutter}}
\label{cap:color-clutter}

Dado que o módulo \textit{Speech to Text} está agora pronto, desenvolveremos um pequeno jogo com ele através do editor \textit{Godot} para demonstrar seu uso e analisar a qualidade do reconhecimento de voz na prática.

Ao longo deste capítulo, relatamos os passos realizados na criação do jogo \textit{Color Clutter}. Para melhor aproveitamento, o leitor precisará do editor \textit{Godot}, na versão 2.1.4, com o módulo instalado. Mencionamos, na seção \ref{modulePublishing}, alguns \textit{links} para baixar uma versão já pronta e para obter instruções de como compilar \textit{Speech to Text} com a \textit{game engine}.

% ---------------------------------------------------------------------

\section{Planejamento}

Além da escrita de código, jogos costumam usar diversos materiais, ou \emph{assets}, como modelos gráficos, música, sons, fontes e texturas. A criação destes demanda bastante tempo e em geral exige uma equipe diversificada e qualificada. Portanto, gostaríamos de produzir um jogo que utilize poucos \textit{assets}; buscaremos quaisquer materiais necessários em páginas Web que os disponibilizem para uso grátis e sem licença.

Quanto à jogabilidade, desejamos colocar ênfase na funcionalidade de reconhecimento de voz; este é o objetivo em sua criação, afinal. O uso de alguns poucos comandos orais em um jogo curto é o ideal para deixar as regras simples, mas com interação razoável com o usuário.

Por fim, definiremos que o jogo e reconhecimento de voz usarão \textbf{inglês}, pois sabemos que um modelo acústico e arquivo de dicionário para \textit{Pocketsphinx} existem e são de uso livre para esta língua.

Com estas características em mente, planejamos o jogo \textbf{\emph{Color Clutter}}.

% ---------------------------------------------------------------------

\subsection{Descrição de \textit{Color Clutter}}

Conforme o nome sugere, \textit{Color Clutter} é um jogo cuja temática envolve uma ``confusão'' entre cores.

Uma típica tela do jogo consiste em um fundo totalmente preenchido com alguma cor \(X\). Em alguma posição da tela, uma outra cor \(Y\) aparece escrita em um tom \(Z\). O objetivo do usuário é falar a cor correta (\(X\), \(Y\) ou \(Z\)), de acordo com o que é solicitado em uma legenda apresentada na tela.

Para acrescentar um aspecto competitivo e tornar \textit{Color Clutter} mais lúdico, o jogo será no formato de rodadas, onde marcaremos quantos acertos o usuário consegue em 1 minuto. Cada resposta correta altera aleatoriamente as cores e a posição da palavra na tela; não há penalidade para uma resposta errada, exceto a perda de tempo acarretada pela mesma.

A figura \ref{color-clutter-screen} apresenta a tela do jogo durante uma rodada. O canto superior esquerdo informa qual cor deve ser pronunciada, enquanto o canto superior direito exibem o tempo restante e o \textit{score} (pontuação) atual do jogador. Na situação apresentada, o usuário precisaria falar a cor correspondente ao \textit{background} (fundo): \textbf{\textit{blue}}.

\begin{figure}[H]
  \centering
  \includegraphics[width=.8\textwidth]{image/color-clutter-screen}
  \caption{Tela do jogo \textit{Color Clutter} durante uma rodada}
  \label{color-clutter-screen}
\end{figure}

% ---------------------------------------------------------------------

\section{Criação do projeto}

% ---------------------------------------------------------------------

\section{Interface do editor \textit{Godot}}

% ---------------------------------------------------------------------

\section{Desenvolvimento}
