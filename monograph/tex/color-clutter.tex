\chapter{Jogo \textit{Color Clutter}}
\label{cap:color-clutter}

Dado que o módulo \textit{Speech to Text} está agora pronto, desenvolveremos um pequeno jogo com ele através do editor \textit{Godot} para demonstrar seu uso e analisar a qualidade do reconhecimento de voz na prática.

Ao longo deste capítulo, relatamos os passos realizados na criação do jogo \textit{Color Clutter}. Para melhor aproveitamento, o leitor precisará do editor \textit{Godot}, na versão 2.1.4, com o módulo instalado. A seção \ref{modulePublishing} indica \textit{links} para baixar uma versão já pronta ou para instruções em como compilar \textit{Speech to Text} com a \textit{game engine}.

% ---------------------------------------------------------------------

\section{Planejamento}

Além da escrita de código, jogos costumam usar diversos materiais, ou \emph{assets}, entre eles modelos gráficos, música, sons, fontes e texturas. A criação destes demanda bastante tempo e em geral exige uma equipe diversificada e qualificada. Portanto, gostaríamos de produzir um jogo que utilize poucos \textit{assets}; buscaremos quaisquer materiais necessários em páginas Web que os disponibilizem para uso grátis e sem licença.

Quanto à jogabilidade, desejamos colocar ênfase na funcionalidade de reconhecimento de voz; este é o objetivo em sua criação, afinal. O uso de alguns poucos comandos orais em um jogo curto é o ideal para deixar as regras simples, mas com interação razoável com o usuário.

Com estas características em mente, planejamos o jogo \textbf{\emph{Color Clutter}}.

% ---------------------------------------------------------------------

\subsection{Descrição de \textit{Color Clutter}}

% ---------------------------------------------------------------------

\section{Criação do projeto}

% ---------------------------------------------------------------------

\section{Interface do editor \textit{Godot}}

% ---------------------------------------------------------------------

\section{Desenvolvimento}
