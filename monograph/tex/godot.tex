\chapter{Godot}
\label{cap:godot}

Voltaremos nossa atenção à \textit{game engine} \textit{Godot} neste capítulo. Em particular, estamos interessados no estudo dos elementos principais que compõe sua arquitetura, a organização do seu código fonte e instruções para compilação.

Para referência, todas as informações relacionadas a \textit{Godot} são referentes à versão \textbf{2.1.4}, que é a mais recente e estável no momento de escrita deste trabalho.

Todas as instruções e comandos apresentados foram originalmente realizados no sistema \texttt{Ubuntu 16.04 LTS, 64-bit} do autor.

% ---------------------------------------------------------------------

\section{História}

O desenvolvimento de \textit{Godot} começou em 2007, através de Juan Linietsky e Ariel Manzur. O nome foi escolhido em homenagem à peça \emph{Waiting for Godot}, de Samuel Beckett, para representar uma biblioteca que cada vez mais ganha novas funcionalidades, mas nunca chegará a um produto definitivo \citep{godotHistory}.

\textit{Godot} é notavelmente conhecido por possuir código aberto, que foi liberado ao público em fevereiro de 2014. Desde então, ganha constantes atualizações para se equiparar a \textit{game engines} competidoras mais sofisticadas, como \textit{Unity} e \textit{Unreal Engine}. Atualmente, encontra-se na versão 2.1.4, com uma versão 3 em beta.

% ---------------------------------------------------------------------

\section{Linguagens}

\textit{Godot} possui seu código fonte escrito primordialmente em \textbf{\textit{C++}}. Apesar de não possuir toda a versatilidade de uma linguagem de \textit{script} como \textit{Python} e \textit{Ruby}, um código escrito em \textit{C++} possui uma execução bem rápida, fator crítico para um software que produzirá jogos. Comparado ao antecessor, \textit{C}, a linguagem oferece ferramentas mais poderosas, como Orientação a Objetos e bibliotecas para tipos abstratos de dados (pilhas, filas, etc.).

Um usuário da \textit{game engine}, no entanto, raramente interage diretamente com \textit{C++}, utilizando-se ao invés disso uma linguagem de \textit{script} nativa ao \textit{Godot} chamada \mbox{\textbf{\textit{GDScript}}}. Por ter uma sintaxe extremamente simples, parecida com \textit{Python}, e ter sido projetada para usufruir da arquitetura da \textit{game engine}, o usuário pode programar classes, estruturas e partes de seu jogo com maior facilidade, utilizando as ferramentas oferecidas pelo software sem precisar se preocupar com detalhes maiores de implementação \citep{godotGDScript}. Em outras palavras, \mbox{\textit{GDScript}} age como uma ``linguagem intermediária'' entre o usuário e as interfaces em \textit{C++} que \textit{Godot} disponibiliza.

% ---------------------------------------------------------------------

\section{Arquitetura}

% ---------------------------------------------------------------------

\subsection{\textit{Node}}

Um \textbf{\textit{node}} (nó) é um dos elementos mais básicos para a criação de jogos em \textit{Godot}. Todo \textit{node} possui um nome, propriedades (que podem ser alteradas/sobrescritas) e um \emph{comportamento}: desenhar um modelo em 3D, mostrar uma interface gráfica, controlar o comportamento de um personagem, etc. \citep{godotNode}.

\textit{Nodes} oferecem a opção de extendê-los, criando um novo tipo de nó com funções adicionais; esta funcionalidade é idêntica à ideia de herança entre classes em Programação Orientada a Objetos.

A propriedade mais importante que oferecem é a adição a outros \textit{nodes}, tornando-se filho deles. Com isso, cria-se uma hierarquia de árvore, deixando claro a dependência de funcionalidades.

Como exemplos, apresentamos a seguir alguns tipos de nós, citando em alguns casos a relação com o código \textit{C++} da \textit{game engine}.

\begin{itemize}
\item \textbf{\textit{BaseButton}}: Oferece funcionalidades básicas a todos os nós do tipo \textit{button} (botões). Internamente (em \textit{C++}), é implementado como uma classe abstrata.

\item \textbf{\textit{Button}}: É um botão padrão, que pode ser clicado pelo usuário. Estende o nó \textit{BaseButton} (na implementação em \textit{C++}, é uma classe que herda de \textit{BaseButton}).

\item \textbf{\textit{Label}}: Apresenta um texto formatado.
\end{itemize}

Suponha que desejamos criar um botão clicável, onde está escrito \textit{``Começar Jogo''}. Ao pensarmos que este botão deverá \emph{conter} um texto, a hierarquia de \textit{nodes} fica intuitiva: iremos criar um nó \textit{Button} que possui um filho \textit{Label}. Além disso, desejamos alterar a propriedade de texto deste último para o valor \textit{``Começar Jogo''}.

Teremos uma visão mais prática de nós na seção \ref{cap:color-clutter}, onde criaremos um jogo em \textit{Godot}.

% ---------------------------------------------------------------------

\subsection{\textit{Scene}}

Uma \textbf{\textit{scene}} (cena) é um grupo de \textit{nodes} organizados em uma hierarquia de árvore. Toda cena possui apenas um nó raiz \citep{godotScene}.

Em \textit{Godot}, executar um jogo é equivalente a executar uma ou mais cenas, que podem ser salvas ou carregadas do disco no decorrer do programa. Ressalta-se que, para o jogo começar, uma \textit{scene} deve ser previamente configurada como a inicial, isto é, a primeira a ser executada quando o jogo é iniciado.

A maior vantagem de um \textit{scene}, portanto, está em sua modularização. Ao invés de se criar um jogo grande com uma quantidade enorme de nós em hierarquia, podem-se fazer várias cenas, com uma \emph{instanciando} outra durante a execução. Tal instância é adicionada na árvore da \textit{scene} que fez a chamada. Quando a cena não é mais necessária, ela pode ser salva no disco, se necessário, e retirada da hierarquia.

% TODO: Colocar alguma imagem de instanciamento

% ---------------------------------------------------------------------

\subsection{\textit{Resource}}

\textbf{\textit{Resources}} (recursos) são outro tipo de dados em \textit{Godot} com uma importância tão grande quanto \textit{nodes}. Todo recurso armazena algum dado, e portanto não realizam uma ação ou processamento por si só \citep{godotResource}.

Uma característica importante de \textit{resources} é que são carregados apenas uma vez do disco. Se um recurso que já está na memória for novamente carregado, será retornado a mesma cópia de antes; em detalhes internos, \textit{Godot} guarda uma referência ao \textit{resource} original.

A classe \textit{Resource} herda, no código C++, de \textit{Reference} (referência). Esta classe e seus filhos possuem uma propriedade especial: a \textit{game engine} se encarrega de automaticamente liberar sua memória alocada quando não são mais referenciadas por nenhuma outra estrutura, trazendo uma característica de \emph{coleta de lixo} ao sistema.

Exemplos de \textit{resources} incluem:

\begin{itemize}
\item \textbf{\textit{Texture}}: Representa uma textura a ser aplicada em um objeto 2D ou 3D.

\item \textbf{\textit{Font}}: Representa uma fonte a ser usada em um texto, interface gráfica, etc.

\item \textbf{\textit{AudioStream}}: Usado para guardar um fluxo de áudio (uma música a ser tocada no jogo, por exemplo).
\end{itemize}

% ---------------------------------------------------------------------

\section{Compilação}

O código fonte de \textit{Godot} está disponível no \textit{GitHub} \citep{godotRepo}. A criação de um módulo, conforme será visto no capítulo \ref{cap:stt-module}, exige a adição de código à \textit{game engine} e sua recompilação, o que justifica a importância dos passos nesta seção.

% ---------------------------------------------------------------------

\subsection{\textit{Scons}}

% ---------------------------------------------------------------------

\subsection{Instruções}
