\begin{frame}{\textit{Pocketsphinx}}{Funcionamento}

\begin{itemize}
\item Desenvolvida pela \textit{Carnegie Mellon University} (projeto \textit{CMUSphinx})

\item Define que palavras são formadas por \textbf{fonemas}

\item<2-> Voz $\rightarrow$ \textit{Utterances} $\rightarrow$ Vetores de características

\item<3-> Utiliza o \textbf{Modelo Oculto de Markov} na interpretação

\begin{itemize}
  \item Trata a fala gravada como uma sequência de estados, que transitam entre si com certa \textbf{probabilidade}
\end{itemize}
\end{itemize}

\uncover<4->{
\begin{center}
\textbf{Estados mais prováveis $\rightarrow$ Melhor interpretação da voz}
\end{center}
}

\end{frame}

% ---------------------------------------------------------------------

\begin{frame}{\textit{Pocketsphinx}}{Arquivos de configuração}

\begin{itemize}
\item \textbf{Modelo acústico}: Arquivos que configuram detectores de fonemas

\item<2-> \textbf{Dicionário fonético}: Mapeamento \{palavras \(\rightarrow\) fonemas\}
\end{itemize}

\uncover<2->{
\begin{center}
  \color{Maroon}\texttt{yellow Y EH L OW}
\end{center}
}

\begin{itemize}
\item<3-> \textbf{Palavras-chave}: Palavras a serem detectadas, de acordo com limiar
\end{itemize}

\uncover<3->{
\begin{center}
  \color{Maroon}\texttt{yellow /1e-6/}
\end{center}
}
\end{frame}
