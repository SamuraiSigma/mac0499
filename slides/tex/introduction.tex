\begin{frame}{Introdução}{Contextualização}

\begin{itemize}
\item Evolução e sofisticação de jogos eletrônicos (\textit{games})

\item Surgimento das \textbf{\emph{game engines}}: \textit{frameworks} voltados para facilitar o desenvolvimento total ou parcial de jogos

\begin{itemize}
  \item Exemplos: \textit{Unreal Engine}, \textit{Unity}, \textit{Godot}
\end{itemize}

\uncover<2->{
\item \textbf{Reconhecimento de voz} vem ficando cada vez mais integrado em nosso dia a dia

\begin{itemize}
  \item Autenticação de usuário, realização de buscas na Internet, etc.
\end{itemize}
}

\vspace{0.5cm}

\uncover<3->{
\textbf{Por que não fazer um trabalho que junte ambos os temas?}
}

\end{itemize}
\end{frame}

% ---------------------------------------------------------------------

\begin{frame}{Introdução}{Objetivo do trabalho}

\begin{block}{Objetivo}
Desenvolver um módulo (\textit{``plugin''}) de reconhecimento de voz para a \textit{game engine} \textit{Godot}, demonstrando depois seu uso com um jogo simples
\end{block}

\vspace{0.5cm}

\uncover<2->{
\textbf{Pergunta}: Por que escolher \textit{Godot}?
}

\uncover<3->{
\textbf{Resposta}: Porque \textit{Godot} é uma \textit{game engine} de \textbf{código aberto}!
}
\end{frame}
