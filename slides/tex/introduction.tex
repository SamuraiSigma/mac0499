\begin{frame}{Introdução}{Contextualização}

\begin{itemize}
\item Evolução e sofisticação de jogos eletrônicos (\textit{games})

\item Surgimento das \textbf{\emph{game engines}}: \textit{frameworks} voltados para facilitar o desenvolvimento total ou parcial de jogos

\begin{itemize}
  \item Exemplos: \textit{Unreal Engine}, \textit{Unity}, \textit{Godot}
\end{itemize}

\uncover<2->{
\item \textbf{Reconhecimento de voz} vem ficando cada vez mais intregado em nosso dia a dia

\begin{itemize}
  \item Autenticação de usuário, realização de buscas na Internet, etc.
\end{itemize}
}

\vspace{0.5cm}

\uncover<3->{
\textbf{Por que não fazer um trabalho que junte ambos os temas?}
}

\end{itemize}
\end{frame}

% ---------------------------------------------------------------------

\begin{frame}{Introdução}{Objetivo do trabalho}

\begin{block}{Objetivo}
Desenvolver um módulo (\textit{``plugin''}) de reconhecimento de voz para a \textit{game engine} \textit{Godot}, demonstrando depois seu uso com um jogo simples
\end{block}

\uncover<2->{
Para atingirmos o objetivo, precisamos:
\begin{enumerate}
\item Estudar a teoria básica por trás de reconhecimento de voz
\item Aprender a usar uma biblioteca que implemente reconhecimento de voz
\item Entender um pouco da arquitetura de \textit{Godot}
\item Implementar o módulo de reconhecimento de voz
\item Criar um simples jogo, usando o módulo, através de \textit{Godot}
\end{enumerate}
}

\end{frame}
